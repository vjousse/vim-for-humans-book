\documentclass{tufte-book}
\usepackage[french]{babel}
\usepackage[utf8]{inputenc}

\hypersetup{colorlinks}% uncomment this line if you prefer colored hyperlinks (e.g., for onscreen viewing)

%%
% Book metadata
\title{Vim - Le guide pour les êtres humains\thanks{Merci à la communauté Vim.}}
\author[Vincent Jousse]{Vincent\ Jousse}
\publisher{Vincent Jousse}

%%
% If they're installed, use Bergamo and Chantilly from www.fontsite.com.
% They're clones of Bembo and Gill Sans, respectively.
%\IfFileExists{bergamo.sty}{\usepackage[osf]{bergamo}}{}% Bembo
%\IfFileExists{chantill.sty}{\usepackage{chantill}}{}% Gill Sans

%\usepackage{microtype}

%%
% Just some sample text
\usepackage{lipsum}

%%
% For nicely typeset tabular material
\usepackage{booktabs}

%%
% For graphics / images
\usepackage{graphicx}
\setkeys{Gin}{width=\linewidth,totalheight=\textheight,keepaspectratio}
\graphicspath{{graphics/}}

% The fancyvrb package lets us customize the formatting of verbatim
% environments.  We use a slightly smaller font.
\usepackage{fancyvrb}
\fvset{fontsize=\normalsize}

%%
% Prints argument within hanging parentheses (i.e., parentheses that take
% up no horizontal space).  Useful in tabular environments.
\newcommand{\hangp}[1]{\makebox[0pt][r]{(}#1\makebox[0pt][l]{)}}

%%
% Prints an asterisk that takes up no horizontal space.
% Useful in tabular environments.
\newcommand{\hangstar}{\makebox[0pt][l]{*}}

%%
% Prints a trailing space in a smart way.
\usepackage{xspace}

% Format code using pygment
\usepackage{minted}
\usemintedstyle{solarized}
%\definecolor{bg}{RGB}{0,43,54}
%Light bg
\definecolor{bg}{RGB}{253,246,227}

%%
% Some shortcuts for Tufte's book titles.  The lowercase commands will
% produce the initials of the book title in italics.  The all-caps commands
% will print out the full title of the book in italics.
\newcommand{\vdqi}{\textit{VDQI}\xspace}
\newcommand{\ei}{\textit{EI}\xspace}
\newcommand{\ve}{\textit{VE}\xspace}
\newcommand{\be}{\textit{BE}\xspace}
\newcommand{\VDQI}{\textit{The Visual Display of Quantitative Information}\xspace}
\newcommand{\EI}{\textit{Envisioning Information}\xspace}
\newcommand{\VE}{\textit{Visual Explanations}\xspace}
\newcommand{\BE}{\textit{Beautiful Evidence}\xspace}

\newcommand{\TL}{Tufte-\LaTeX\xspace}

% Prints the month name (e.g., January) and the year (e.g., 2008)
\newcommand{\monthyear}{%
  \ifcase\month\or January\or February\or March\or April\or May\or June\or
  July\or August\or September\or October\or November\or
  December\fi\space\number\year
}

% Prints an epigraph and speaker in sans serif, all-caps type.
\newcommand{\openepigraph}[2]{%
  %\sffamily\fontsize{14}{16}\selectfont
  \begin{fullwidth}
  \sffamily\large
  \begin{doublespace}
  \noindent\allcaps{#1}\\% epigraph
  \noindent\allcaps{#2}% author
  \end{doublespace}
  \end{fullwidth}
}

% Inserts a blank page
\newcommand{\blankpage}{\newpage\hbox{}\thispagestyle{empty}\newpage}

\usepackage{units}

% Typesets the font size, leading, and measure in the form of 10/12x26 pc.
\newcommand{\measure}[3]{#1/#2$\times$\unit[#3]{pc}}

% Macros for typesetting the documentation
\newcommand{\hlred}[1]{\textcolor{Maroon}{#1}}% prints in red
\newcommand{\hangleft}[1]{\makebox[0pt][r]{#1}}
\newcommand{\hairsp}{\hspace{1pt}}% hair space
\newcommand{\hquad}{\hskip0.5em\relax}% half quad space
\newcommand{\TODO}{\textcolor{red}{\bf TODO!}\xspace}
\newcommand{\ie}{\textit{i.\hairsp{}e.}\xspace}
\newcommand{\eg}{\textit{e.\hairsp{}g.}\xspace}
\newcommand{\na}{\quad--}% used in tables for N/A cells
\providecommand{\XeLaTeX}{X\lower.5ex\hbox{\kern-0.15em\reflectbox{E}}\kern-0.1em\LaTeX}
\newcommand{\tXeLaTeX}{\XeLaTeX\index{XeLaTeX@\protect\XeLaTeX}}
% \index{\texttt{\textbackslash xyz}@\hangleft{\texttt{\textbackslash}}\texttt{xyz}}
\newcommand{\tuftebs}{\symbol{'134}}% a backslash in tt type in OT1/T1
\newcommand{\doccmdnoindex}[2][]{\texttt{\tuftebs#2}}% command name -- adds backslash automatically (and doesn't add cmd to the index)
\newcommand{\doccmddef}[2][]{%
  \hlred{\texttt{\tuftebs#2}}\label{cmd:#2}%
  \ifthenelse{\isempty{#1}}%
    {% add the command to the index
      \index{#2 command@\protect\hangleft{\texttt{\tuftebs}}\texttt{#2}}% command name
    }%
    {% add the command and package to the index
      \index{#2 command@\protect\hangleft{\texttt{\tuftebs}}\texttt{#2} (\texttt{#1} package)}% command name
      \index{#1 package@\texttt{#1} package}\index{packages!#1@\texttt{#1}}% package name
    }%
}% command name -- adds backslash automatically
\newcommand{\doccmd}[2][]{%
  \texttt{\tuftebs#2}%
  \ifthenelse{\isempty{#1}}%
    {% add the command to the index
      \index{#2 command@\protect\hangleft{\texttt{\tuftebs}}\texttt{#2}}% command name
    }%
    {% add the command and package to the index
      \index{#2 command@\protect\hangleft{\texttt{\tuftebs}}\texttt{#2} (\texttt{#1} package)}% command name
      \index{#1 package@\texttt{#1} package}\index{packages!#1@\texttt{#1}}% package name
    }%
}% command name -- adds backslash automatically
\newcommand{\docopt}[1]{\ensuremath{\langle}\textrm{\textit{#1}}\ensuremath{\rangle}}% optional command argument
\newcommand{\docarg}[1]{\textrm{\textit{#1}}}% (required) command argument
\newenvironment{docspec}{\begin{quotation}\ttfamily\parskip0pt\parindent0pt\ignorespaces}{\end{quotation}}% command specification environment
\newcommand{\docenv}[1]{\texttt{#1}\index{#1 environment@\texttt{#1} environment}\index{environments!#1@\texttt{#1}}}% environment name
\newcommand{\docenvdef}[1]{\hlred{\texttt{#1}}\label{env:#1}\index{#1 environment@\texttt{#1} environment}\index{environments!#1@\texttt{#1}}}% environment name
\newcommand{\docpkg}[1]{\texttt{#1}\index{#1 package@\texttt{#1} package}\index{packages!#1@\texttt{#1}}}% package name
\newcommand{\doccls}[1]{\texttt{#1}}% document class name
\newcommand{\docclsopt}[1]{\texttt{#1}\index{#1 class option@\texttt{#1} class option}\index{class options!#1@\texttt{#1}}}% document class option name
\newcommand{\docclsoptdef}[1]{\hlred{\texttt{#1}}\label{clsopt:#1}\index{#1 class option@\texttt{#1} class option}\index{class options!#1@\texttt{#1}}}% document class option name defined
\newcommand{\docmsg}[2]{\bigskip\begin{fullwidth}\noindent\ttfamily#1\end{fullwidth}\medskip\par\noindent#2}
\newcommand{\docfilehook}[2]{\texttt{#1}\index{file hooks!#2}\index{#1@\texttt{#1}}}
\newcommand{\doccounter}[1]{\texttt{#1}\index{#1 counter@\texttt{#1} counter}}

% Keys shortcuts
\newcommand{\tesc}{\hlred{Esc} (\hlred{Échap})\xspace}
\newcommand{\ttesc}{la touche \tesc}
\newcommand{\ti}{\hlred{i}\xspace}
\newcommand{\tti}{la touche \ti}
\newcommand{\tto}{la touche \hlred{o}\xspace}

\newcommand{\vimscmd}[1]{\colorbox{bg}{\hlred{\Verb|{\footnotesize #1}|}}}
\newcommand{\vimcmd}[1]{\colorbox{bg}{\hlred{\Verb|#1|}}}

\newcommand{\vim}{\emph{Vim}\xspace}
\newcommand{\vimrc}{\emph{.vimrc}\xspace}


% Generates the index
\usepackage{makeidx}
\makeindex


\begin{document}

% Front matter
\frontmatter

% r.1 blank page
\blankpage

% r.3 full title page
\maketitle


% v.4 copyright page
\newpage
\begin{fullwidth}
~\vfill
\thispagestyle{empty}
\setlength{\parindent}{0pt}
\setlength{\parskip}{\baselineskip}
Copyright \copyright\ \the\year\ \thanklessauthor

\par\smallcaps{Publié par \thanklesspublisher}

\par\smallcaps{Style \LaTeX{} \url{http://tufte-latex.googlecode.com}}

\par Si vous n'avez pas payé cette copie, bah tant pis pour moi ;) Mais sachez que vous auriez dû !

\end{fullwidth}

% r.5 contents
\tableofcontents

\listoffigures

\listoftables

% r.7 dedication
\cleardoublepage
~\vfill
\begin{doublespace}
\noindent\fontsize{18}{22}\selectfont\itshape
\nohyphenation
Merci à ma femme et mes enfants qui ont permis à ce livre de voir le jour.
\end{doublespace}
\vfill
\vfill

% r.9 introduction
\cleardoublepage

\chapter*{Introduction}

\newthought{Lorsque le besoin} d'écrire ou de coder se fait se sentir, le choix d'un éditeur de texte est primordial. Il en existe énormément sur le "marché", mais peu d'entre eux peuvent se targuer d'environ 40 ans d'existence. C'est le cas d'\emph{Emacs}\sidenote{\url{http://www.gnu.org/software/emacs/}}, de \emph{Vi} et de son "successeur" \emph{ViM}\sidenote{\url{http://www.vim.org/}}. Ils ont été créés dans les années 70 et sont toujours très utilisés actuellement\sidenote{À noter que \emph{ViM} n'est arrivé qu'en 1991}. Comme vous avez sans doute pu le voir, ce n'est pas grâce à la beauté de leur site internet ou à l'efficacité de leur communication. Voici quelques raisons de leur succès :

\begin{description}
    \item[Pour la vie] \hfill \\ Ils s'apprennent une fois et s'utilisent pour toujours. Dans un monde où les technologies/langages changent tout le temps, c'est une aubaine de pouvoir investir sur du long terme.
    \item[Partout] \hfill \\ Ils sont disponibles sur toutes les plateformes possibles et imaginables (et l'ont toujours été).
    \item[Enlarge votre productivité] \hfill \\ Ils présentent un rapport temps investi / gain de productivité fabuleux.
    \item[Couteaux Suisses] \hfill \\ Ils permettent d'éditer tout et n'importe quoi. Quand vous changerez de langage de programmation, vous n'aurez pas à changer d'éditeur. À noter que ce livre a bien sur été écrit avec \emph{ViM}.
\end{description}

Et pourtant, ils restent difficiles à apprendre. Non pas qu'ils soient plus compliqués qu'autre chose, non pas que vous ne soyez pas à la hauteur, mais plutôt à cause d'un manque de pédagogie des différentes documentations.

Ce livre a pour but de pallier à ce manque en vous guidant tout au long de votre découverte de \emph{ViM}\sidenote{Je laisse \emph{Emacs} à ceux qui savent. Pour un bref comparatif c'est ici : \url{http://fr.wikipedia.org/wiki/Guerre_d'éditeurs}. Les goûts et les couleurs …}.

Si vous aussi vous en avez marre d'attendre la release de TextMate 2 ? D'essayer le n-ième éditeur à la mode et de devoir tout réapprendre et ce jusqu'à la prochaine mode ? De devoir changer d'éditeur quand vous passez de votre Mac, à votre Linux, à votre serveur ? Vous aussi, rejoignez la communauté des gens heureux de leur éditeur de texte. Le changement, c'est maintenant.

\section{Pour qui ?}

Toute personne étant amenée à produire du texte (code, livre, rapports, présentations, ...) de manière régulière. Les développeurs sont bien sur une cible privilégiée, mais pas uniquement.

Par exemple vous êtes :
\begin{description}
    \item[Étudiant] Si vous voulez faire bien sur un CV, c'est un must (en plus d'être un attrape geekette en puissance\footnote{À confirmer.}). Vous aurez un outil unique pour écrire tout ce que vous avez à écrire (et que vous pourrez réutiliser tout au long de votre carrière) : vos rapports en \LaTeX, vos présentations, votre code (si vous avez besoin d'OpenOffice ou de Word pour écrire vos rapports, il est temps de changer d'outil et d'utiliser \LaTeX).
    \item[Prof] Il est temps de montrer l'exemple et d'apprendre à vos étudiants à bien utiliser un des outils qui leur servira à vie, bien plus qu'un quelconque langage de programmation.
    \item[Codeur] Investir dans votre outil de tous les jours est indispensable. Quitte à apprendre des raccourcis claviers, autant le faire de manière utile. Si cet investissement est encore rentable dans 10 ans, c'est juste l'investissement parfait, c'est \emph{Vim}.
    \item[Administrateur système Unix] Si vous utilisez emacs vous êtes pardonnable, si vous utilisez nano/pico vous êtes pendable, sinon il est grand temps de s'y mettre les gars, c'est un des cas d'utilisation parfait (un éditeur de texte surpuissant ne nécessitant pas d'interface graphique).
    \item[Écrivain] Si vous écrivez en markdown/RST/WikiMarkup ou en \LaTeX, \emph{Vim} vous fera gagner beaucoup de temps. Vous ne pourrez plus repasser à un autre éditeur, ou vous voudrez le "Vimifier" à tout prix.
\end{description}

Faites moi confiance, je suis passé et repassé par ces 5 rôles, mon meilleur investissement a toujours été \emph{Vim}, et de loin.

\section{Ce que vous apprendrez (entre autres choses)}

\begin{itemize}
    \item Comment utiliser \emph{Vim} comme un éditeur "normal" d'abord (vous savez, ceux qui permettent d'ouvrir des fichiers, de cliquer avec la souris, qui ont une coloration syntaxique ...). En somme, la démystification de \emph{Vim} qui vous permettra d'aller plus loin.
    \item Comment passer de l'édition de texte classique à la puissance de \emph{Vim}, petit à petit (c'est là que l'addiction commence).
    \item Comment vous passer de la souris et pourquoi c'est la meilleure chose qu'il puisse vous arriver quand vous programmez/tapez du texte.
    \item Comment vous pouvez facilement déduire les "raccourcis claviers" avec quelques règles simples.
\end{itemize}

Si je devais le résumer en une phrase : puisque vous vous considérez comme {\bf un artiste, passez du temps à apprendre votre outil}, comme un artiste, une bonne fois pour toute.

\section{Ce que vous n'apprendrez pas (entre autres choses)}

\begin{itemize}
    \item Vous n'apprendrez pas comment installer/configurer {\em Vim} pour Windows. Pas que ce ne soit pas faisable, mais je n'ai que très peu de connaissances sous Windows. Ça viendra peut-être, mais pas tout de suite. On couvrira ici Linux/Unix (et donc Mac Os X aussi).
    \item Vous n'apprendrez pas comment utiliser \emph{Vi} (notez l'absence du "m"). Je vais vous apprendre à être productif pour coder/produire du texte avec \emph{Vim}, pas à faire le beau devant les copains avec \emph{Vi}. Pour ceux qui ne suivent pas, \emph{Vi} est "l'ancêtre de \emph{ViM} (qui veut dire \emph{Vi} - \emph{IMproved}, \emph{Vi} amélioré)" et est installé par défaut sur tous les Unix (même sur votre Mac OS X).
    \item Vous n'apprendrez pas à connaitre les entrailles de \emph{Vim} par c\oe ur : ce n'est pas une référence, mais un guide utile et pragmatique.
    \item Vous n'apprendrez pas comment modifier votre \emph{Vim} parce que vous préférez le rouge au bleu : je vous ferai utiliser le thème [solarized](http://ethanschoonover.com/solarized), il est juste parfait pour travailler.
\end{itemize}

\section{Le plus dur, c'est de commencer (et de continuer à commencer)}

Alors, prêt pour l'aventure ? Prêt à sacrifier une heure pour débuter avec \emph{Vim}, une semaine pour devenir familier avec la bête, et le reste de votre vie pour vous féliciter de votre choix ? Alors c'est parti !


%%
% Start the main matter (normal chapters)


\mainmatter

\chapter{Rendre \vim utilisable}

\newthought{Ça peut paraître étonnant} comme approche, mais c'est pour moi la première chose à faire : rendre \vim utilisable par un humain lambda. Si tout le monde semble s'accorder sur le fait que \vim est un \textbf{éditeur très puissant}, tout le monde pourra aussi s'accorder sur le fait que de base, il est juste \textbf{imbitable}. Soyons honnête, sans une configuration par défaut minimale, utiliser \vim est \textbf{contre-productif}. 

C'est à mon avis le premier obstacle à surmonter avant toute autre chose. C'est ce que les autres éditeurs ``Mainstream'' comme Textmate, Sublimetext, Notepad++ ou Netbeans proposent, c'est à dire un environnement à minima utilisable tel quel, même si l'on en n'exploite pas la totalité.

Voici donc ce qui manque à un \vim nu (et ce qui est pour moi une \textbf{cause d'abandon pour beaucoup} d'entre vous) :

\begin{marginfigure}%
  \includegraphics[width=\linewidth]{solarized-yinyang.png}
  \caption{Le thème \emph{Solarized} en sombre et en clair. \url{http://ethanschoonover.com/solarized}}
  \label{fig:solarized}
\end{marginfigure}

\begin{description}
    \item[Configuration par défaut] \vim est configurable grâce à un fichier nommé \emph{vimrc}, qui est bien entendu vide par défaut. La première étape va être d'avoir un fichier \emph{vimrc} avec une configuration minimale.
    \item[Coloration syntaxique] De base, \vim est tout blanc et tout moche. On va utiliser le thème \emph{Solarized} (\url{http://ethanschoonover.com/solarized}). Si votre but est d'être efficace, c'est le meilleur thème disponible actuellement (tout éditeur de texte confondu). La figure \ref{fig:solarized} vous donne une idée des deux look disponibles (clair ou sombre). Pour ma part j'utilise le thème sombre.
    \item[Explorateur de fichiers] Si vous utilisez \vim avec une interface graphique (ce qui est le cas de 99\% d'entre vous je suppose) vous avez par défaut un menu \Verb|Fichier| vous permettant d'ouvrir un fichier. C'est certes un bon début, mais avoir à disposition un explorateur de projet à la Netbeans ou à la Textmate peut s'avérer très pratique. Pour obtenir le même comportement, nous utiliserons Nerdtree. À savoir qu'à la fin de ce livre, vous n'aurez plus besoin de la souris (et donc des menus et autres boutons).
\end{description}

Ce chapitre est indispensable si vous n'avez que peu d'expérience (voir pas du tout) avec \vim. À la fin de ce chapitre, vous aurez un \vim dont vous pourrez commencer à vous servir pour vos tâches de tous les jours. Cela devrait être suffisant pour vous permettre d'apprendre le reste petit à petit. Car il n'y a pas de secret, il vous faudra pratiquer pour apprendre \vim, alors autant commencer de suite et le moins douloureusement possible.

En revanche, si vous êtes déjà familier avec \vim et n'utilisez déjà plus la souris, vous pouvez sagement sauter ce chapitre (soyez sur tout de même de donner sa chance au thème \emph{Solarized}).

\section{Préambule indispensable : le mode insertion}

Prenons le pari de créer le fichier \emph{vimrc} avec \vim lui même. Comme je vous le disais, le plus tôt vous commencerez, le mieux ce sera.
Vous devrez certainement commencer par installer une version de \vim. Si vous utilisez un Mac, essayez MacVim \sidenote{MacVim: \url{http://code.google.com/p/macvim/}} sans aucune hésitation. Si vous utilisez GNU/Linux ou tout autre système ``Unix'' vous devriez surement avoir gVim à votre disposition (ou tout du moins facilement installable grâce à votre gestionnaire de logiciels). Pour Windows, il semblerait y avoir une version disponible sure le site officiel de \vim\sidenote{Page de téléchargement officielle de \vim : \url{http://www.vim.org/download.php}}, mais je ne l'ai pas testée.

Cliquez sur \Verb|Fichier (File) -> Nouveau (New)|. Le texte d'accueil par défaut de \vim devrait avoir disparu et vous devriez avoir une page blanche comme sur la figure \ref{fig:vim-new}. 

\begin{figure}%
  \includegraphics[width=\linewidth]{vim-new.png}
  \caption{Nouveau fichier vide.}
  \label{fig:vim-new}
\end{figure}

Commençons par entrer un commentaire dans d'entête du fichier pour y mentionner notre nom. Pour pouvoir entrer du texte appuyez sur \tti (le curseur devrait changer d'aspect) et entrez le commentaire ci-dessous\sidenote{Si vous ne savez pas trop ce que vous avez fait et que \vim vous affiche des trucs en rouge en bas à gauche au ne semble pas réagir comme il faut quand vous appuyez sur \tti, appuyez plusieurs fois sur \ttesc, ça devrait vous remettre au mode par défaut de \vim}.
\begin{listing}[H]

    \begin{minted}[bgcolor=bg, gobble=8]{vim}
        " VIM Configuration - Vincent Jousse
    \end{minted}
    \caption{Votre première ligne avec \vim.}
    \label{code:first-comment}
\end{listing}

Vous aurez remarqué que les commentaires en \emph{VimL} (le langage de configuration de \vim) commencent par un \Verb|"|. Appuyez ensuite sur \ttesc pour revenir au mode par défaut (le mode normal) de \vim. Et voilà le travail, cf figure \ref{fig:vim-first-comment}.

\begin{figure}%
  \includegraphics[width=\linewidth]{vim-first-comment.png}
  \caption{Mon premier commentaire.}
  \label{fig:vim-first-comment}
\end{figure}

Tout ça pour ça me direz vous, et vous avez bien raison. Mais tout cela a une logique que je vais vous expliquer. L'avantage de \vim est qu'il est généralement logique, quand vous avez compris la logique, tout vous semble limpide et tomber sous le sens.

Par défaut, \vim est lancé dans un mode que l'on appelle le mode ``Normal''. C'est à dire que ce mode n'est pas fait pour écrire du texte (ça, ça sera le mode ``Insert'') mais juste pour se déplacer et manipuler du texte. C'est la présence de ces 2 différents modes (il y en a d'autres mais ce n'est pas le sujet pour l'instant) qui fait toute la puissance de \vim. Il vous faudra un certain temps pour vous rendre compte de cette puissance par vous même, alors faites moi juste confiance pour l'instant.

Si vous vous demandez pourquoi ces modes, pourquoi on semble se compliquer la vie (on se la simplifie en fait) et en quel honneur, dans le mode par défaut, il n'est même pas possible d'insérer du texte, lisez attentivement la section qui suit.

\section{Les modes : d'où \vim tire sa puissance}

\begin{figure}%
  \includegraphics[width=\linewidth]{hand-position.png}
  \caption{Position de repos, clavier QWERTY. \emph{Illustration par Cy21 - CC-BY-SA-3.0 (\url{www.creativecommons.org/licenses/by-sa/3.0}) ou GFDL (\url{www.gnu.org/copyleft/fdl.html}), via Wikimedia Commons \url{http://commons.wikimedia.org/wiki/File:Typing-home-keys-hand-position.svg}}}
  \label{fig:hand-position}
\end{figure}

\section{La configuration par défaut : indispensable}



\section{La coloration syntaxique : le futile indispensable}

\TODO

\section{L'explorateur de fichiers : parce qu'ouvrir des fichiers, ça peut être pratique}


\begin{fullwidth}
    \begin{minted}[bgcolor=bg]{vim}
        " VIM Configuration - Vincent Jousse
        " Some links : http://nvie.com/posts/how-i-boosted-my-vim/

        set nocompatible                  " Must come first because it changes other options.

        " Use pathogen to easily modify the runtime path to include all
        " plugins under the ~/.vim/bundle directory
        call pathogen#helptags()
        call pathogen#runtime_append_all_bundles()

        " Quickly edit/reload the vimrc file
        nmap <silent> <leader>ev :e $MYVIMRC<CR>
        nmap <silent> <leader>sv :so $MYVIMRC<C

        " Colorsheme
        set t_Co=256
    \end{minted}
\end{fullwidth}

\TODO




\chapter{\vim : L'outil de manipulation de texte rêvé}

Ce qui fait et fera encore le succès de \vim est sa capacité à faciliter les manipulations de texte. Certes il va vous proposer des fonctionnalités propres à chaque tâche que vous effectuerez \footnote{Souvent par l'intermédiaire de plugin.} comme la validation syntaxique de code, la correction orthographique, \ldots Mais à la fin, c'est toujours à écrire/corriger/manipuler/se déplacer dans du texte que vous passerez la majeure partie de votre temps. 

C'est la que l'approche de \vim est différente d'IDE comme Eclipse / Netbeans / PhpStorm/\ldots Là où ces IDE vont mettre l'accent sur les particularités de votre langage de programmation tout en vous fournissant des capacités de manipulation de texte basiques, \vim adopte l'approche opposée : vous serez très efficace à manipuler/écrire du texte quelque soit le texte et vous pourrez enrichir \vim avec des fonctionnalités propres à votre langage de programmation via des.

Nous allons donc voir dans ce chapitre comment utiliser \vim à bon escient (vous allez commencer à oublier votre souris) et quelle est la logique derrière tout cet enchaînement de commandes qui paraissent barbare au non initié. Vous devriez pouvoir, à la fin de ce chapitre, vous passer de votre souris pour éditer/manipuler le contenu d'un fichier.

\chapter{Les plugins indispensables}

\printindex

\end{document}
