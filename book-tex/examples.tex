\chapter{Pense-bête et exemples}

Nous venons de faire un tour d'horizon de tout ce qui est nécessaire pour bien commencer dans la vie avec \vim. Tout cela devrait être suffisant pour pouvoir l'utiliser au quotidien. C'est le secret de la réussite avec \vim : réussir à l'encrer dans nos habitudes journalières. Une fois que cela est fait, le reste devrait couler de source.

Cette dernière partie est là pour vous donner un endroit de référence où vous pourrez revenir comme bon vous semble lorsque vous serez un peu perdu sur comment faire telle ou telle chose avec \vim. Ce chapitre est composé de deux parties. La première est un ensemble de questions réponses qui couvre les principaux problèmes que les débutants rencontrent lorsqu'ils commencent. Le but est de répondre aux questions du type : « rha mais comment on fait ça, c'était pourtant si simple avec mon ancien éditeur ». La seconde partie est une liste (non exhaustive) des commandes \vim les plus utiles dont vous pourrez vous servir comme pense-bête. Allez hop, au boulot.

\section{Questions / réponses}

\subsection{Comment quitter \vim ?}

La première chose à faire est de se mettre en mode normal. Grosso modo, excitez-vous sur \ttesc ou \ttsemicolon en fonction de votre configuration et vous devriez vous retrouver en mode normal. Ensuite tapez \vimcmd{:q} pour quitter. Il y a de grandes chances que \vim ne vous laisse pas faire. Si vous avez des modifications non enregistrées par exemple, il ne voudra pas quitter. Vous pouvez annuler les modification en le forçant à quitter grâce à l'utilisation de \vimcmd{!} comme ceci : \vimcmd{:q!}. Vous pouvez aussi enregistrer vos modifications puis quitter comme ceci : \vimcmd{:wq}.

\subsection{Comment sauvegarder sous ?}

En mode normal, si vous tapez \vimcmd{:w}, \vim par défaut sauvegarde vos modifications dans le fichier courant. Si vous souhaitez utiliser un autre nom de fichier pour « sauvegarder sous », vous avez juste à lui spécifier le nom du fichier après \vimcmd{w} comme ceci : \vimcmd{:w monfichier.txt}. \vim sauvegardera alors votre fichier sous le nom \emph{monfichier.txt}. En revanche \vim n'ouvrira pas \emph{monfichier.txt}, il restera sur votre précédent fichier.

Si vous souhaitez que \vim sauvegarde sous \emph{monfichier.txt} et ouvre ensuite ce fichier dans le tampon courant, vous devrez utiliser \vimcmd{:sav monfichier.txt}.

\subsection{Comment copier/couper coller ?}

Celle là est facile, j'y ai déjà consacré un chapitre, cf. \nameref{sec:se-deplacer}. 

En résumé : 

\begin{itemize}
    \item Passez en mode visuel avec \ttv,
    \item Sélectionnez ce que vous voulez copier en vous déplaçant,
    \item Copiez avec \tty\xspace ou couper avec \ttx ou \ttd,
    \item Collez après l'emplacement du curseur avec \ttp ou avant l'emplacement du curseur avec \ttP.
\end{itemize}

\section{Pense-bête}

\bigskip
\begin{tabular}[H]{|c|c|}
  \hline
  Résultat attendu & Action\\
  \hline
  Sauvegarder & :w (w pour write)\\
  Sauvegarder et quitter & :wq (wq pour write and quit)\\
  Quitter sans sauvegarder (forcer à quitter) & :q! \\
  Annuler (Undo) & u \\
  Refaire (Redo) & Ctrl+r \\
  \hline
\end{tabular}
\bigskip

