\chapter{Pense-bête et exemples}

Nous venons de faire un tour d'horizon de tout ce qui est nécessaire pour bien commencer dans la vie avec \vim. Tout cela devrait être suffisant pour pouvoir l'utiliser au quotidien. C'est le secret de la réussite avec \vim : réussir à l'encrer dans nos habitudes journalières. Une fois que cela est fait, le reste devrait couler de source.

Cette dernière partie est là pour vous donner un endroit de référence où vous pourrez revenir comme bon vous semble lorsque vous serez un peu perdu sur comment faire telle ou telle chose avec \vim. Ce chapitre est composé de deux parties. La première est un ensemble de questions réponses qui couvre les principaux problèmes que les débutants rencontrent lorsqu'ils commencent. Le but est de répondre aux questions du type : « rha mais comment on fait ça, c'était pourtant si simple avec mon ancien éditeur ». La seconde partie est une liste (non exhaustive) des commandes \vim les plus utiles dont vous pourrez vous servir comme pense-bête. Allez hop, au boulot.

\section{Questions / réponses}

\section{Pense-bête}

\bigskip
\begin{tabular}[H]{|c|c|}
  \hline
  Résultat attendu & Action\\
  \hline
  Sauvegarder & :w (w pour write)\\
  Sauvegarder et quitter & :wq (wq pour write and quit)\\
  Quitter sans sauvegarder (forcer à quitter) & :q! \\
  Annuler (Undo) & u \\
  Refaire (Redo) & Ctrl+r \\
  \hline
\end{tabular}
\bigskip

